\documentclass{Z}
%% need no \usepackage{Sweave}

\author{Achim Zeileis\\Wirtschaftsuniversit\"at Wien \And
        Gabor Grothendieck}
\Plainauthor{Achim Zeileis, Gabor Grothendieck}

\title{\pkg{zoo}: An \proglang{S3} Class and Methods for
  Indexed Totally Ordered Observations}
\Plaintitle{zoo: An S3 Class and Methods for
  Indexed Totally Ordered Observations}
%% \Shorttitle{\pkg{zoo}: \proglang{S3} Infrastructure for Indexed Totally Ordered Observations}

\Keywords{totally ordered observations, irregular time series,
  \proglang{S3}, \proglang{R}}
\Plainkeywords{totally ordered observations, irregular time series, S3, R}

\Abstract{
  \pkg{zoo} is an \proglang{R} package providing an \proglang{S3}
  class with methods for indexed totally ordered observations, such as
  irregular time series. Its key design goals are independence of a
  particular index/time/date class and consistency with base
  \proglang{R} and the \code{"ts"} class for
  regular time series. This paper describes how these are achieved
  within \pkg{zoo} and provides several illustrations 
  of the available methods for \code{"zoo"} objects which include
  plotting, merging and binding, several mathematical operations,
  extracting and replacing data and index, coercion and \code{NA}
  handling.
}

\begin{document}


%\VignetteIndexEntry{zoo: An S3 Class and Methods for Indexed Totally Ordered Observations}
%\VignetteDepends{zoo,tseries,fBasics,strucchange,quantreg,DAAG}
%\VignetteKeywords{totally ordered observations, irregular time series, S3, R}
%\VignettePackage{zoo}



\section{Introduction} \label{sec:intro}

The \proglang{R} system for statistical computing
\citep[\url{http://www.R-project.org/}]{zoo:R:2004}
ships with a a class for regularly spaced time series,
\code{"ts"} in package \pkg{stats}, but has no native class for
irregularly spaced time series. With the increased interest in
computational finance with \proglang{R} over the last years
several implementations of classes for irregular time series 
emerged which are aimed particularly at finance applications.
These include the \proglang{S3} classes \code{"timeSeries"}
in package \pkg{fBasics} from the \pkg{Rmetrics} bundle \citep{zoo:fBasics:2004}
and \code{"irts"} in package \pkg{tseries} \citep{zoo:tseries:2004}
and the \proglang{S4} class \code{"its"} in package \pkg{its} \citep{zoo:its:2004}.
With these packages available, why would anybody want yet another 
package providing infrastructure for irregular time series?
The above mentioned implementations have in common that they are restricted to a particular
class for the time scale: the former implementation comes with its own time class
\code{"timeDate"} built on top of the \code{"POSIXt"} classes
available in base \proglang{R} whereas the latter two use \code{"POSIXct"} directly.
And this was the starting point for the \pkg{zoo} project: the first author
of the present paper needed
more general support for ordered observations, independent of a particular
index class, for the package \pkg{strucchange}
\citep{zoo:Zeileis+Leisch+Hornik:2002}. Hence the package was called
\pkg{zoo} which stands for \underline{Z}'s \underline{o}rdered \underline{o}bservations.
Since the first release, a major part of the additions to \pkg{zoo}
were provided by the second author of this paper, so that the name
of the package does not really reflect the authorship anymore.
Nevertheless, independence of a particular index class remained
the most important design goal. While the package evolved to its current
status, a second key design goal became more and more clear: to provide
methods to standard generic functions for the \code{"zoo"} class that 
are similar to those for the \code{"ts"} class (and base \proglang{R} in
general) such that the usage of \pkg{zoo} is rather intuitive because
few additional commands have to be learned.
This paper describes how these design goals are implemented in \pkg{zoo}.
The resulting package provides the \code{"zoo"} class which offers an
extensive (and still growing) set of standard and new methods for working
on indexed observations and `talks' to the classes \code{"ts"}, \code{"its"},
\code{"irts"} and \code{"timeSeries"}.

The remainder of the paper is organized as follows:
Section~\ref{sec:zoo-class} explains how \code{"zoo"} objects are created
and illustrates how the corresponding methods for plotting, merging and
binding, several mathematical operations, extracting and replacing data
and index, coercion and \code{NA} handling can be used. Section~\ref{sec:combining}
outlines how other packages can build on this basic infrastructure.
Section~\ref{sec:summary} gives a few summarizing remarks and an outlook
on future developments. Finally, an appendix provides a reference card that
gives an overview of the functionality contained in \pkg{zoo}.

\section[The class "zoo" and its methods]{The class \code{"zoo"} and its methods}
\label{sec:zoo-class}


\subsection[Creation of "zoo" objects]{Creation of \code{"zoo"} objects}
\label{sec:zoo}

The simple idea for the creation of \code{"zoo"} objects is to have
some vector or matrix of observations \code{x} which are totally ordered
by some index vector. In time series applications, this index is a measure of
time but every other numeric, character or even more abstract vector that
provides a total ordering of the observations is also suitable. Objects
of class \code{"zoo"} are created by the function
\begin{Scode}
zoo(x, order.by)
\end{Scode}
where \code{x} is the vector or matrix of observations\footnote{In principle,
more general objects can be indexed, but currently \pkg{zoo} does not support this.
Development plans are that \pkg{zoo} should eventually support indexed factors,
data frames and lists.} and \code{order.by}
is the index by which the observations should be ordered. It has to be
of the same length as \code{NROW(x)}, i.e., either the same length as \code{x}
for vectors or the same number of rows for matrices. (This constraint is
not imposed for zero length vectors.)   The \code{"zoo"} object
created is essentially the vector/matrix as before but has an additional
\code{"index"} attribute in which the index is stored.\footnote{There is some
limited support for indexed factors available in which case the \code{"zoo"}
object also has an attribute \code{"oclass"} with the original class
of \code{x}. This feature is still under development and might change in future
versions.} Both the observations in the vector/matrix \code{x}
and the index \code{order.by} can, in principle, be of arbitrary classes. However, most of the
following methods (plotting, aggregating, mathematical operations) for \code{"zoo"}
objects are typically only useful for numeric observations \code{x}. Special
effort in the design was put into independence from a particular class for
the index vector. In \pkg{zoo}, it is assumed that combination \code{c()},
querying the \code{length()}, value matching \code{MATCH()}, subsetting \code{[,},
and, of course, ordering \code{ORDER()} work when applied to the index. 
This is the case, e.g., for standard numeric and character vectors and for
vectors of classes \code{"Date"}, \code{"POSIXct"} or \code{"times"}
from package \pkg{chron}, but not for the class \code{"dateTime"} in \pkg{fBasics}.
In the last case, the solution is to provide methods for the above mentioned
functions so that indexing \code{"zoo"} objects with \code{"dateTime"} vectors works
(see Section~\ref{sec:fBasics} for an example).
To achieve this  independence of the index class, new generic functions for
ordering (\code{ORDER()}) and value matching (\code{MATCH()}) are introduced
as the corresponding base functions \code{order()} and \code{match()} are 
non-generic. The default methods simply call the corresponding base functions, i.e.,
no new method needs to be introduced for a particular index class if the 
non-generic functions \code{order()} and \code{match()} work for this class.

To illustrate the usage of \code{zoo}, we first load the package and set the
random seed to make the examples in this paper exactly reproducible.

\begin{Schunk}
\begin{Sinput}
> library(zoo)
> set.seed(1071)
\end{Sinput}
\end{Schunk}

Then, we create two vectors \code{z1} and \code{z2} with \code{"POSIXct"} 
indexes, one with random observations
\begin{Schunk}
\begin{Sinput}
> z1.index <- ISOdatetime(2004, rep(1:2, 5), sample(28, 10), 0, 
+     0, 0)
> z1.data <- rnorm(10)
> z1 <- zoo(z1.data, z1.index)
\end{Sinput}
\end{Schunk}
and one with a sine wave
\begin{Schunk}
\begin{Sinput}
> z2.index <- as.POSIXct(paste(2004, rep(1:2, 5), sample(1:28, 
+     10), sep = "-"))
> z2.data <- sin(2 * 1:10/pi)
> z2 <- zoo(z2.data, z2.index)
\end{Sinput}
\end{Schunk}
Furthermore, we create a matrix \code{Z} with random observations and a \code{"Date"}
index
\begin{Schunk}
\begin{Sinput}
> Z.index <- as.Date(sample(12450:12500, 10))
> Z.data <- matrix(rnorm(30), ncol = 3)
> colnames(Z.data) <- c("Aa", "Bb", "Cc")
> Z <- zoo(Z.data, Z.index)
\end{Sinput}
\end{Schunk}
In the examples above, the generation of indexes looks a bit awkward
due to the fact the indexes need to be randomly generated (and there 
are no special functions for random indexes because these is rarely 
needed in practice). In ``real world'' applications, the indexes
are typically part of the raw data set read into \proglang{R} so the
code would be even simpler. See Section~\ref{sec:combining}
for such examples.\footnote{Note, that in the code above a new \code{as.Date}
method, provided in \pkg{zoo}, is used to convert days 
since 1970-01-01 to class \code{"Date"}. See the respective help page 
for more details.}

Methods to several standard generic functions are available for
\code{"zoo"} objects, such as \code{print}, \code{summary}, \code{str}, \code{head},
\code{tail} and \code{[} (subsetting), a few of which are illustrated in
the following. 

There are three printing code styles for \code{"zoo"} objects: vectors are by default
printed in \code{"horizontal"} style
\begin{Schunk}
\begin{Sinput}
> z1
\end{Sinput}
\begin{Soutput}
 2004-01-05  2004-01-14  2004-01-19  2004-01-25  2004-01-27  2004-02-07 
 0.74675994  0.02107873 -0.29823529  0.68625772  1.94078850  1.27384445 
 2004-02-12  2004-02-16  2004-02-20  2004-02-24 
 0.22170438 -2.07607585 -1.78439244 -0.19533304 
\end{Soutput}
\begin{Sinput}
> z1[3:7]
\end{Sinput}
\begin{Soutput}
2004-01-19 2004-01-25 2004-01-27 2004-02-07 2004-02-12 
-0.2982353  0.6862577  1.9407885  1.2738445  0.2217044 
\end{Soutput}
\end{Schunk}
and matrices in \code{"vertical"} style
\begin{Schunk}
\begin{Sinput}
> Z
\end{Sinput}
\begin{Soutput}
           Aa          Bb          Cc         
2004-02-02  1.25543390  0.68157316 -0.63292049
2004-02-08 -1.49458326  1.32341223 -1.49442269
2004-02-09 -1.87462247 -0.87329289  0.62733971
2004-02-21 -0.14538608  0.45234903 -0.14597401
2004-02-22  0.22542418  0.53838938  0.23136133
2004-02-29  1.20695518  0.31814222 -0.01129202
2004-03-05 -1.20861025  1.42379785 -0.81614483
2004-03-10 -0.11039563  1.34774254  0.95522468
2004-03-14  0.84202385 -2.73842019  0.23150695
2004-03-20 -0.19019104  0.12308872 -1.51862157
\end{Soutput}
\begin{Sinput}
> Z[1:3, 2:3]
\end{Sinput}
\begin{Soutput}
           Bb         Cc        
2004-02-02  0.6815732 -0.6329205
2004-02-08  1.3234122 -1.4944227
2004-02-09 -0.8732929  0.6273397
\end{Soutput}
\end{Schunk}
Additionally, there is a \code{"plain"} style which simply first prints the data 
and then the index.

Summaries and most other methods for \code{"zoo"} objects are carried out
column wise, reflecting the rectangular structure. In addition,
a summary of the index is provided.

\begin{Schunk}
\begin{Sinput}
> summary(z1)
\end{Sinput}
\begin{Soutput}
     Index                           z1          
 Min.   :2004-01-05 00:00:00   Min.   :-2.07608  
 1st Qu.:2004-01-20 12:00:00   1st Qu.:-0.27251  
 Median :2004-02-01 12:00:00   Median : 0.12139  
 Mean   :2004-02-01 09:36:00   Mean   : 0.05364  
 3rd Qu.:2004-02-15 00:00:00   3rd Qu.: 0.73163  
 Max.   :2004-02-24 00:00:00   Max.   : 1.94079  
\end{Soutput}
\begin{Sinput}
> summary(Z)
\end{Sinput}
\begin{Soutput}
     Index                  Aa                Bb                Cc          
 Min.   :2004-02-02   Min.   :-1.8746   Min.   :-2.7384   Min.   :-1.51862  
 1st Qu.:2004-02-12   1st Qu.:-0.9540   1st Qu.: 0.1719   1st Qu.:-0.77034  
 Median :2004-02-25   Median :-0.1279   Median : 0.4954   Median :-0.07863  
 Mean   :2004-02-25   Mean   :-0.1494   Mean   : 0.2597   Mean   :-0.25739  
 3rd Qu.:2004-03-08   3rd Qu.: 0.6879   3rd Qu.: 1.1630   3rd Qu.: 0.23147  
 Max.   :2004-03-20   Max.   : 1.2554   Max.   : 1.4238   Max.   : 0.95522  
\end{Soutput}
\end{Schunk}

 
\subsection{Plotting}
\label{sec:plot}

The \code{plot} method for \code{"zoo"} objects, in particular for
multivariate \code{"zoo"} series, is based on the corresponding
method for (multivariate) regular time series. It relies on \code{plot}
and \code{lines} methods being available for the index class which can
plot the index against the observations.

By default the \code{plot} method creates a panel for each series
\begin{Schunk}
\begin{Sinput}
> plot(Z)
\end{Sinput}
\end{Schunk}
but can also display all series in a single panel
\begin{Schunk}
\begin{Sinput}
> plot(Z, plot.type = "single", col = 2:4)
\end{Sinput}
\end{Schunk}
In both cases additional graphical parameters like color \code{col},
plotting character \code{pch} and line type \code{lty} can be
expanded to the number of series. But the \code{plot} method for
\code{"zoo"} objects offers some more flexibility in specification
of graphical parameters as in
\begin{Schunk}
\begin{Sinput}
> plot(Z, type = "b", lty = 1:3, pch = list(Aa = 1:5, Bb = 2, Cc = 4), 
+     col = list(Bb = 2, 4))
\end{Sinput}
\end{Schunk}
The argument \code{lty} behaves as before and sets every series in another
line type. The \code{pch} argument is a named list that assigns to each series
a different vector of plotting characters each of which is expanded to the 
number of observations. Such a list does not necessarily have to include the names of all
series, but can also specify a subset. For the remaining series the default parameter
is then used which can again be changed: e.g., in the above example the \code{col} argument
is set to display the series \code{"Bb"} in red and all remaining series in blue.
The results of the multiple panel plots are depicted in Figure~\ref{fig:plot13} and the
single panel plot in \ref{fig:plot2}.

\begin{figure}[tbh]
\begin{center}
\includegraphics{zoo-plot2-repeat}
\caption{\label{fig:plot2} Example of a single panel plot}
\end{center}
\end{figure}


\begin{figure}[p]
\begin{center}
\includegraphics{zoo-plot1-repeat}
\includegraphics{zoo-plot3-repeat}
\caption{\label{fig:plot13} Examples of multiple panel plots}
\end{center}
\end{figure}


\subsection{Merging and binding}
\label{sec:merge}

As for many rectangular data formats in \proglang{R}, there are
both methods for combining the rows and columns of \code{"zoo"}
objects respectively. For the \code{rbind} method the number of
columns of the combined objects has to be identical and the
indexes may not overlap.
\begin{Schunk}
\begin{Sinput}
> rbind(z1[5:10], z1[2:3])
\end{Sinput}
\begin{Soutput}
 2004-01-14  2004-01-19  2004-01-27  2004-02-07  2004-02-12  2004-02-16 
 0.02107873 -0.29823529  1.94078850  1.27384445  0.22170438 -2.07607585 
 2004-02-20  2004-02-24 
-1.78439244 -0.19533304 
\end{Soutput}
\end{Schunk}
The \code{cbind} method by default combines the columns by the union of
the indexes and fills the created gaps by \code{NA}s.\footnote{Note,
that \code{cbind} currently is inferior to \code{merge} regarding the
column naming of the resulting object.}
\begin{Schunk}
\begin{Sinput}
> cbind(z1, z2)
\end{Sinput}
\begin{Soutput}
           z1          z2         
2004-01-03          NA  0.94306673
2004-01-05  0.74675994 -0.04149429
2004-01-14  0.02107873          NA
2004-01-17          NA  0.59448077
2004-01-19 -0.29823529 -0.52575918
2004-01-24          NA -0.96739776
2004-01-25  0.68625772          NA
2004-01-27  1.94078850          NA
2004-02-07  1.27384445          NA
2004-02-08          NA  0.95605566
2004-02-12  0.22170438 -0.62733473
2004-02-13          NA -0.92845336
2004-02-16 -2.07607585          NA
2004-02-20 -1.78439244          NA
2004-02-24 -0.19533304          NA
2004-02-25          NA  0.56060280
2004-02-26          NA  0.08291711
\end{Soutput}
\end{Schunk}
In fact, the \code{cbind} method is synonymous with the \code{merge}
method except that the latter provides additional arguments
which allow for combining the columns by the intersection
of the indexes using the argument \code{all = FALSE}
\begin{Schunk}
\begin{Sinput}
> merge(z1, z2, all = FALSE)
\end{Sinput}
\begin{Soutput}
           z1          z2         
2004-01-05  0.74675994 -0.04149429
2004-01-19 -0.29823529 -0.52575918
2004-02-12  0.22170438 -0.62733473
\end{Soutput}
\end{Schunk}
Additionally, the filling pattern can be changed in \code{merge},
the naming of the
columns can be modified and the return class of the result can
be specified. In the case of merging of objects with 
different index classes, \proglang{R} gives a warning and tries to
coerce the indexes. Merging objects with different index classes is
generally discouraged---if it is used nevertheless, it is the
responsibility of the user to ensure that the result is as intended.

Another function which performs operations along a subset of indexes
is \code{aggregate}, which is discussed in this section although
it does not combine several objects. Using the \code{aggregate} method, \code{"zoo"} objects
are split into subsets along a coarser index grid,
summary statistics are computed for each and then the 
reduced object is returned. In the following example,
first a function is set up which returns for a given \code{"Date"}
value the corresponding first of the month. This function is then
used to compute the coarser grid for the \code{aggregate} call: in
the first example the mean of the observations in the month
is returned, in the second example the first observation.
%%FIXME: maybe also firstofquarter, see man pages

\begin{Schunk}
\begin{Sinput}
> firstofmonth <- function(x) as.Date(sub("..$", "01", format(x)))
> aggregate(Z, firstofmonth(Z.index), mean)
\end{Sinput}
\begin{Soutput}
           Aa          Bb          Cc         
2004-02-01  0.53820841  0.04508597 -0.12412352
2004-03-01 -1.18080051  0.58156655 -0.45730045
\end{Soutput}
\begin{Sinput}
> aggregate(Z, firstofmonth(Z.index), head, 1)
\end{Sinput}
\begin{Soutput}
           Aa         Bb         Cc        
2004-02-01  1.2554339  0.6815732 -0.6329205
2004-03-01 -1.4945833  1.3234122 -1.4944227
\end{Soutput}
\end{Schunk}


\subsection{Mathematical operations}
\label{sec:Ops}

To allow for standard mathematical operations among \code{"zoo"}
objects, \pkg{zoo} extends group generic functions \code{Ops}.
These perform the operations only for the intersection of the
indexes of the objects. As an example, the summation and logical
comparison with $<$ of \code{z1} and \code{z2} yield
\begin{Schunk}
\begin{Sinput}
> z1 + z2
\end{Sinput}
\begin{Soutput}
2004-01-05 2004-01-19 2004-02-12 
 0.7052657 -0.8239945 -0.4056304 
\end{Soutput}
\begin{Sinput}
> z1 < z2
\end{Sinput}
\begin{Soutput}
2004-01-05 2004-01-19 2004-02-12 
     FALSE      FALSE      FALSE 
\end{Soutput}
\end{Schunk}

Additionally, methods for transposing \code{t} of \code{"zoo"}
objects---which coerces to a matrix before, see below---and 
computing cumulative quantities such as
\code{cumsum}, \code{cumprod}, \code{cummin}, \code{cummax}
which are all applied column wise.
\begin{Schunk}
\begin{Sinput}
> cumsum(Z)
\end{Sinput}
\begin{Soutput}
           Aa         Bb         Cc        
2004-02-02  1.2554339  0.6815732 -0.6329205
2004-02-08 -0.2391494  2.0049854 -2.1273432
2004-02-09 -2.1137718  1.1316925 -1.5000035
2004-02-21 -2.2591579  1.5840415 -1.6459775
2004-02-22 -2.0337337  2.1224309 -1.4146162
2004-02-29 -0.8267785  2.4405731 -1.4259082
2004-03-05 -2.0353888  3.8643710 -2.2420530
2004-03-10 -2.1457844  5.2121135 -1.2868283
2004-03-14 -1.3037606  2.4736933 -1.0553214
2004-03-20 -1.4939516  2.5967820 -2.5739429
\end{Soutput}
\end{Schunk}


\subsection{Extracting and replacing the data and the index}
\label{sec:window}

\pkg{zoo} provides several generic functions and methods
to work on the data contained in a \code{"zoo"} object, the
index (or time) attribute associated to it, and on both data and
index.

The data stored in \code{"zoo"} objects can be extracted by
\code{coredata} which strips off all \code{"zoo"}-specific attributes and 
it can be replaced using \code{coredata<-}. Both are new generic
functions\footnote{The \code{coredata} functionality is similar in spirit to the \code{core}
function in \pkg{its} and \code{value} in \pkg{tseries}. However, the 
focus of those functions is somewhat narrower and we try to provide 
more general purpose generic functions. See the respective manual
page for more details.}
with methods for \code{"zoo"} objects as illustrated in the following
example.
\begin{Schunk}
\begin{Sinput}
> coredata(z1)
\end{Sinput}
\begin{Soutput}
 [1]  0.74675994  0.02107873 -0.29823529  0.68625772  1.94078850  1.27384445
 [7]  0.22170438 -2.07607585 -1.78439244 -0.19533304
\end{Soutput}
\begin{Sinput}
> coredata(z1) <- 1:10
> z1
\end{Sinput}
\begin{Soutput}
2004-01-05 2004-01-14 2004-01-19 2004-01-25 2004-01-27 2004-02-07 2004-02-12 
         1          2          3          4          5          6          7 
2004-02-16 2004-02-20 2004-02-24 
         8          9         10 
\end{Soutput}
\end{Schunk}

The index associated with a \code{"zoo"} object can be extracted
by \code{index} and modified by \code{index<-}. As the interpretation
of the index as ``time'' in time series applications is natural,
there are also synonymous methods \code{time} and \code{time<-}. 
Hence, the commands \code{index(z2)} and \code{time(z2)}
return equivalent results.
\begin{Schunk}
\begin{Sinput}
> index(z2)
\end{Sinput}
\begin{Soutput}
 [1] "2004-01-03 CET" "2004-01-05 CET" "2004-01-17 CET" "2004-01-19 CET"
 [5] "2004-01-24 CET" "2004-02-08 CET" "2004-02-12 CET" "2004-02-13 CET"
 [9] "2004-02-25 CET" "2004-02-26 CET"
\end{Soutput}
\end{Schunk}
The index scale of \code{z2} can be changed to that of \code{z1} by
\begin{Schunk}
\begin{Sinput}
> index(z2) <- index(z1)
> z2
\end{Sinput}
\begin{Soutput}
 2004-01-05  2004-01-14  2004-01-19  2004-01-25  2004-01-27  2004-02-07 
 0.94306673 -0.04149429  0.59448077 -0.52575918 -0.96739776  0.95605566 
 2004-02-12  2004-02-16  2004-02-20  2004-02-24 
-0.62733473 -0.92845336  0.56060280  0.08291711 
\end{Soutput}
\end{Schunk}

The start and the end of the index/time vector can be queried by
\code{start} and \code{end}:
\begin{Schunk}
\begin{Sinput}
> start(z1)
\end{Sinput}
\begin{Soutput}
[1] "2004-01-05 CET"
\end{Soutput}
\begin{Sinput}
> end(z1)
\end{Sinput}
\begin{Soutput}
[1] "2004-02-24 CET"
\end{Soutput}
\end{Schunk}


To work on both data and index/time, \pkg{zoo} provides
\code{window} and \code{window<-} methods for \code{"zoo"} objects.
In both cases the window is specified by
\begin{Scode}
window(x, index, start, end)
\end{Scode}
where \code{x} is the \code{"zoo"} object, \code{index} is a set
of indexes to be selected (by default the full index of \code{x})
and \code{start} and \code{end} can be used to restrict the 
\code{index} set. 
\begin{Schunk}
\begin{Sinput}
> window(Z, start = as.Date("2004-03-01"))
\end{Sinput}
\begin{Soutput}
           Aa         Bb         Cc        
2004-03-05 -1.2086102  1.4237978 -0.8161448
2004-03-10 -0.1103956  1.3477425  0.9552247
2004-03-14  0.8420238 -2.7384202  0.2315069
2004-03-20 -0.1901910  0.1230887 -1.5186216
\end{Soutput}
\begin{Sinput}
> window(Z, index = index(Z)[5:8], end = as.Date("2004-03-01"))
\end{Sinput}
\begin{Soutput}
           Aa          Bb          Cc         
2004-02-22  0.22542418  0.53838938  0.23136133
2004-02-29  1.20695518  0.31814222 -0.01129202
\end{Soutput}
\end{Schunk}

The first example selects all observations starting from 2004-03-01
whereas the second selects from the from the 5th to 8th observation
those up to 2004-03-01.

The same syntax can be used for the corresponding replacement function.
\begin{Schunk}
\begin{Sinput}
> window(z1, end = as.POSIXct("2004-02-01")) <- 9:5
> z1
\end{Sinput}
\begin{Soutput}
2004-01-05 2004-01-14 2004-01-19 2004-01-25 2004-01-27 2004-02-07 2004-02-12 
         9          8          7          6          5          6          7 
2004-02-16 2004-02-20 2004-02-24 
         8          9         10 
\end{Soutput}
\end{Schunk}

Two methods that are standard in time series applications
are \code{lag} and \code{diff}. These are available with the same
arguments as the \code{"ts"} methods.\footnote{\code{diff} also
has an additional argument that also allows for geometric and
not only allows arithmetic differences. Furthermore, note the sign
of the lag in \code{lag}: by default it is positive and shifts the 
observations \emph{forward}, to obtain the more standard \emph{backward}
shift the lag has to be negative.}

\begin{Schunk}
\begin{Sinput}
> lag(z1, k = -1)
\end{Sinput}
\begin{Soutput}
2004-01-14 2004-01-19 2004-01-25 2004-01-27 2004-02-07 2004-02-12 2004-02-16 
         9          8          7          6          5          6          7 
2004-02-20 2004-02-24 
         8          9 
\end{Soutput}
\begin{Sinput}
> merge(z1, lag(z1, k = 1))
\end{Sinput}
\begin{Soutput}
           z1 lag(z1, k = 1)
2004-01-05  9  8            
2004-01-14  8  7            
2004-01-19  7  6            
2004-01-25  6  5            
2004-01-27  5  6            
2004-02-07  6  7            
2004-02-12  7  8            
2004-02-16  8  9            
2004-02-20  9 10            
2004-02-24 10 NA            
\end{Soutput}
\begin{Sinput}
> diff(z1)
\end{Sinput}
\begin{Soutput}
2004-01-14 2004-01-19 2004-01-25 2004-01-27 2004-02-07 2004-02-12 2004-02-16 
        -1         -1         -1         -1          1          1          1 
2004-02-20 2004-02-24 
         1          1 
\end{Soutput}
\end{Schunk}



\subsection[Coercion to and from "zoo"]{Coercion to and from \code{"zoo"}}
\label{sec:as.zoo}

Coercion to and from \code{"zoo"} objects is available for objects of
various classes, in particular \code{"ts"}, \code{"irts"} and \code{"its"}
objects can be coerced to \code{"zoo"} using the  respective \code{as.zoo}
method. The reverse coercion is available for \code{"its"} and for \code{"irts"}
(the latter in package \code{tseries}).
Furthermore, \code{"zoo"} objects can be coerced to vectors, matrices, lists and
data frames (the latter dropping the index/time attribute). A simple example is
\begin{Schunk}
\begin{Sinput}
> as.data.frame(Z)
\end{Sinput}
\begin{Soutput}
           Aa         Bb          Cc
1   1.2554339  0.6815732 -0.63292049
2  -1.4945833  1.3234122 -1.49442269
3  -1.8746225 -0.8732929  0.62733971
4  -0.1453861  0.4523490 -0.14597401
5   0.2254242  0.5383894  0.23136133
6   1.2069552  0.3181422 -0.01129202
7  -1.2086102  1.4237978 -0.81614483
8  -0.1103956  1.3477425  0.95522468
9   0.8420238 -2.7384202  0.23150695
10 -0.1901910  0.1230887 -1.51862157
\end{Soutput}
\end{Schunk}


\subsection[NA handling]{\code{NA} handling}
\label{sec:NA}

Four methods for dealing with \code{NA}s (missing observations) 
in the observations are applicable to \code{"zoo"} objects:
\code{na.omit}, \code{na.contiguous}, \code{na.approx} and \code{na.locf}.
\code{na.omit}---or its default method to be more precise---returns a \code{"zoo"}
object with incomplete observations removed. \code{na.contiguous}
extracts the longest consecutive stretch of non-missing values.
This function is currently made generic in \pkg{zoo}
with a \code{"zoo"} method and the \pkg{stats} function
as the default.\footnote{\code{na.contiguous}
will be generic in base \proglang{R} from version 2.1.0 on.}
Furthermore, new generic functions
\code{na.approx} and \code{na.locf} and corresponding default methods are introduced in \pkg{zoo}.
The former replaces \code{NA}s by linear interpolation (using the
function \code{approx}) and the name of the latter
stands for \underline{l}ast \underline{o}bservation \underline{c}arried
\underline{f}orward. It replaces missing observations by the most recent
non-\code{NA} prior to it. Leading \code{NA}s, which cannot be replaced
by precious observations, are removed in both functions by default.

\begin{Schunk}
\begin{Sinput}
> z1[sample(1:10, 3)] <- NA
> z1
\end{Sinput}
\begin{Soutput}
2004-01-05 2004-01-14 2004-01-19 2004-01-25 2004-01-27 2004-02-07 2004-02-12 
         9         NA          7          6          5          6         NA 
2004-02-16 2004-02-20 2004-02-24 
         8          9         NA 
\end{Soutput}
\begin{Sinput}
> na.omit(z1)
\end{Sinput}
\begin{Soutput}
2004-01-05 2004-01-19 2004-01-25 2004-01-27 2004-02-07 2004-02-16 2004-02-20 
         9          7          6          5          6          8          9 
\end{Soutput}
\begin{Sinput}
> na.contiguous(z1)
\end{Sinput}
\begin{Soutput}
2004-01-19 2004-01-25 2004-01-27 2004-02-07 
         7          6          5          6 
\end{Soutput}
\begin{Sinput}
> na.approx(z1)
\end{Sinput}
\begin{Soutput}
2004-01-05 2004-01-14 2004-01-19 2004-01-25 2004-01-27 2004-02-07 2004-02-12 
  9.000000   7.714286   7.000000   6.000000   5.000000   6.000000   7.111111 
2004-02-16 2004-02-20 
  8.000000   9.000000 
\end{Soutput}
\begin{Sinput}
> na.approx(z1, 1:NROW(z1))
\end{Sinput}
\begin{Soutput}
2004-01-05 2004-01-14 2004-01-19 2004-01-25 2004-01-27 2004-02-07 2004-02-12 
         9          8          7          6          5          6          7 
2004-02-16 2004-02-20 
         8          9 
\end{Soutput}
\begin{Sinput}
> na.locf(z1)
\end{Sinput}
\begin{Soutput}
2004-01-05 2004-01-14 2004-01-19 2004-01-25 2004-01-27 2004-02-07 2004-02-12 
         9          9          7          6          5          6          6 
2004-02-16 2004-02-20 2004-02-24 
         8          9          9 
\end{Soutput}
\end{Schunk}

As the above example illustrates, \code{na.approx} uses by default
the underlying time scale for interpolation. This can be changed, e.g.,
to an equidistant spacing, by setting the second argument of
\code{na.approx}.

\section[Combining zoo with other packages]{Combining \pkg{zoo} with other packages}
\label{sec:combining}

The main purpose of the package \pkg{zoo} is to provide basic infrastructure for
working with indexed totally ordered observations that can be either employed by
users directly or can be a basic ingredient on top of which other packages can
build. The latter is illustrated with a few brief examples involving the packages
\pkg{strucchange}, \pkg{tseries}, \pkg{fBasics} and \pkg{stats} in this section.

\subsection[strucchange: Empirical fluctuation processes]{\pkg{strucchange}: Empirical fluctuation processes}
\label{sec:strucchange}

The package \pkg{strucchange} provides a collection of methods for testing,
monitoring and dating structural changes, in particular in linear regression models.
Tests for structural change assess whether the parameters of a model remain
constant over an ordering with respect to a specified variable, usually time.
To adequatly store and visualize empirical fluctuation processes which 
capture instabilities over this ordering, a data type for indexed ordered
observations is required. This was the motivation for starting the \pkg{zoo}
project.

A simple example for the need of \code{"zoo"} objects in \pkg{strucchange}
which is not (easily) be implemented by other irregular time series classes
available on CRAN is described in the following. We assess the constancy of the
electrical resistance over the apparent juice content of kiwi fruits.\footnote{A different
approach would be to test whether the slope of a regression of electrical resistance
on juice content changes with increasing juice content, i.e., to test for
instabilities in \code{ohms ~ juice} instead of \code{ohms ~ 1}. Both lead to 
similar results.} The data
set \code{fruitohms} is contained in the \pkg{DAAG} package \citep{zoo:DAAG:2004}.
The fitted \code{ocus} object contains the OLS-based CUSUM process for the mean
of the electrical resistance (variable \code{ohms}) indexed by the juice
content (variable \code{juice}).

\begin{Schunk}
\begin{Sinput}
> library(strucchange)
> library(DAAG)
\end{Sinput}
\begin{Soutput}
Loading required package: leaps 
Loading required package: oz 
\end{Soutput}
\begin{Sinput}
> data(fruitohms)
> ocus <- gefp(ohms ~ 1, order.by = ~juice, data = fruitohms)
\end{Sinput}
\end{Schunk}

This OLS-based CUSUM process can be visualized using the \code{plot} method
for \code{"gefp"} objects which builds on the \code{"zoo"} method and yields in
this case the plot in Figure~\ref{fig:strucchange} showing the process which
crosses its 5\% critical value and 
thus signals a significant decrease in the mean electrical resistance over the
juice content. for more information on the package \pkg{strucchange} and the 
function \code{gefp} see \cite{zoo:Zeileis+Leisch+Hornik:2002} and 
\cite{zoo:Zeileis:2004}.

\begin{figure}
\begin{center}
\begin{Schunk}
\begin{Sinput}
> plot(ocus)
\end{Sinput}
\end{Schunk}
\includegraphics{zoo-strucchange2}
\caption{\label{fig:strucchange} Empirical M-fluctuation process for \code{fruitohms} data}
\end{center}
\end{figure}


\subsection[tseries: Historical financial data]{\pkg{tseries}: Historical financial data}
\label{sec:tseries}

A typical application for irregular time series which became increasingly
important over the last years in computational statistics and finance is
daily (or higher frequent) financial data. The package \pkg{tseries} provides
the function \code{get.hist.quote} for obtaining historical financial data
by querying Yahoo! Finance at \url{http://finance.yahoo.com/},
an online portal quoting data provided by Reuters. The following code
queries the quotes of Lucent Technologies starting from 2001-01-01:

\begin{Schunk}
\begin{Sinput}
> library(tseries)
> LU <- get.hist.quote(instrument = "LU", start = "2001-01-01", 
+     end = "2004-09-30", origin = "1970-01-01")
\end{Sinput}
\end{Schunk}

\begin{Schunk}
\begin{Soutput}
time series starts 2001-01-02
\end{Soutput}
\end{Schunk}

In the returned \code{LU} object the irregular data is stored by extending
it in a regular grid and filling the gaps with \code{NA}s. The time is stored
in days starting from an \code{origin}, in this case specified to be 1970-01-01, the
origin used by the \code{Date} class.
This series can be transformed easily into an irregular \code{"zoo"} series 
using a \code{"Date"} index. The log-difference returns for Lucent 
Technologies is depicted in Figure~\ref{fig:tseries}.

\begin{Schunk}
\begin{Sinput}
> LU <- as.zoo(LU)
> index(LU) <- as.Date(index(LU))
> LU <- na.omit(LU)
\end{Sinput}
\end{Schunk}

\begin{figure}
\begin{center}
\begin{Schunk}
\begin{Sinput}
> plot(diff(log(LU)))
\end{Sinput}
\end{Schunk}
\includegraphics{zoo-tseries3}
\caption{\label{fig:tseries} Log-difference returns for Lucent Technologies}
\end{center}
\end{figure}


\subsection[fBasics: Indexes of class "timeDate"]{\pkg{fBasics}: Indexes of class \code{"timeDate"}}
\label{sec:fBasics}

Although the methods in \pkg{zoo} work out of the box for many index classes,
it might be necessary for some index classes to provide \code{c}, \code{length},
\code{ORDER} and \code{MATCH} methods such that the methods in \pkg{zoo} 
work properly. An example for such an index class which requires a bit more
attention is \code{"timeDate"} from the \pkg{fBasics} package.

But after the necessary methods have been defined
\begin{Schunk}
\begin{Sinput}
> length.timeDate <- function(x) prod(x@Dim)
> ORDER.timeDate <- function(x, ...) order(as.POSIXct(x), ...)
> MATCH.timeDate <- function(x, table, nomatch = NA, ...) match(as.POSIXct(x), 
+     as.POSIXct(table), nomatch = NA, ...)
\end{Sinput}
\end{Schunk}
the class \code{"timeDate"} can be used for indexing \code{"zoo"} objects.
The following example illustrates how \code{z2} can be transformed
to use the \code{"timeDate"} class.
\begin{Schunk}
\begin{Sinput}
> library(fBasics)
> z2td <- zoo(coredata(z2), timeDate(index(z2), FinCenter = "GMT"))
> z2td
\end{Sinput}
\begin{Soutput}
 2004-01-05  2004-01-14  2004-01-19  2004-01-25  2004-01-27  2004-02-07 
 0.94306673 -0.04149429  0.59448077 -0.52575918 -0.96739776  0.95605566 
 2004-02-12  2004-02-16  2004-02-20  2004-02-24 
-0.62733473 -0.92845336  0.56060280  0.08291711 
\end{Soutput}
\end{Schunk}

\subsection[stats: (Dynamic) regression modelling]{\pkg{stats}: (Dynamic) regression modelling}
\label{sec:stats}

\code{zoo} provides a facility for extending regression functions such
as \code{lm} to handle time series.  One simply encloses the \code{formula}
argument in \code{I(...)} and ensures that all variables in
the formula are of class \code{"zoo"} or all are of class \code{"ts"}.

Basic regression functions, like \code{lm} or \code{glm}, in which regression
relationships are specified via a \code{formula} only have limited
support for time series regression. The reason is that \code{lm(formula, ...)}
calls the generic function \code{model.frame(formula, ...)} to create a 
a data frame with the variables required. This dispatches to \code{model.frame.formula}
which does not deal specifically with (various types of) time series data.
Therefore, it would be desirable to dispatch to a specialized \code{model.frame}
method depending on the type of the dependent variable. As this is a non-standard
dispatch, \pkg{zoo} provides the following mechanism: In the call to the regression 
function, the \code{formula} is insulated by \code{I()}, e.g., as in
\code{lm(I(formula), ...)}, leaving \code{formula} unaltered but returning an object
of class \code{"AsIs"}. Then, \code{model.frame.AsIs} is called which examines the
dependent variable of the \code{formula} and then dispatches to \code{model.frame.foo}
if this is of class \code{"foo"}. In \pkg{zoo}, the methods \code{model.frame.zoo}
and \code{model.frame.ts} are provided which are able to create model frames
from formulas in which \emph{all} variables are of class \code{"zoo"} or \code{"ts"},
respectively. The advantage of \code{model.frame.zoo} is that it aligns
the variables along a common index, it allows the usage of \code{lag} and
\code{diff} in the model specification and works with the \code{NA} handling methods
described in Section~\ref{sec:NA}. Therefore, dynamic linear regression models
can be fit easily using the standard \code{lm} function by just insulating
\code{I(formula)} in the corresponding call\footnote{In addition to \code{lm}
and \code{glm}, this approach works for many other regression functions including
\code{randomForest} ensembles from \pkg{randomForest},
\code{svm} support vector machines from \pkg{e1071},
\code{lqs} resistant regression from \pkg{MASS},
\code{nnet} neural networks from \pkg{nnet},
\code{rq} quantile regression from \pkg{quantreg},
and possibly many others.}.

A simple example based on artificial data is given below: the lag of a dependent
variable is explained by the first differences of a numeric regressor and an
explanatory factor. Note, that the variables have different indexes. First, a linear
regression model is fitted, then a quantile regression is carried out for the same
equation.

\begin{Schunk}
\begin{Sinput}
> yz <- zoo(1:20)^2
> xz <- zoo(1:18)^2
> fz <- zoo(gl(4, 5))
> lm(I(lag(yz) ~ diff(xz) + fz))
\end{Sinput}
\begin{Soutput}
Call:
lm(formula = I(lag(yz) ~ diff(xz) + fz))

Coefficients:
(Intercept)     diff(xz)           fz  
    -68.275       10.092        8.922  
\end{Soutput}
\begin{Sinput}
> library(quantreg)
> rq(I(lag(yz) ~ diff(xz) + fz))
\end{Sinput}
\begin{Soutput}
Call:
rq(formula = I(lag(yz) ~ diff(xz) + fz))

Coefficients:
(Intercept)    diff(xz)          fz 
  -58.50000    11.64286   -10.28571 

Degrees of freedom: 17 total; 14 residual
\end{Soutput}
\end{Schunk}

See the help page of \code{model.frame.zoo} for more examples
and additional information. Furthermore, note that this feature is under 
development and might subject to changes in future versions.



\section{Summary and outlook} \label{sec:summary}

The package \pkg{zoo} provides an \proglang{S3} class and methods
for indexed totally ordered observations, such as irregular time series.
Its key design goals are independence of a particular index class 
and compatibility with standard generics similar to the behaviour of 
the corresponding \code{"ts"} methods. This paper describes how
these are implemented in \pkg{zoo} and illustrates the usage of 
the methods for plotting, merging and
binding, several mathematical operations, extracting and replacing data
and index, coercion and \code{NA} handling.

An indexed object of class \code{"zoo"} can be thought of as data plus index
where the data are essentially vectors or matrices and the index can be
a vector of (in principle) arbitrary class. Therefore, objects of classes
\code{"ts"}, \code{"its"}, \code{"irts"} and \code{"timeSeries"} can easily
be transformed into \code{"zoo"} objects---the reverse transformation is also possible 
provided that the index fulfills the restrictions of the respective class.
Hence, the \code{"zoo"} class can also be used as the basis for other
classes of indexed and objects and more specific functionality can be built on
top of it.

Whereas a lot of effort was put into achieving independence of a particular
index class, the types of data that can be indexed with \code{"zoo"} are currently
limited to vectors and matrices, typically containing numeric values. Although,
there is some limited support available for indexed factors, one important 
direction for future development of \pkg{zoo} is to add better support for other
objects that can also naturally be indexed including specifically factors, data
frames and lists.



\section*{Computational details}

The results in this paper were obtained using \proglang{R}
2.0.0 with the packages
\pkg{zoo} 0.9--1,
\pkg{strucchange} 1.2--7,
\pkg{fBasics} 200.10058,
\pkg{tseries} 0.9--24,
\pkg{randomForest} 4.5--1 and
\pkg{DAAG} 0.37.
\proglang{R} itself and all packages used are available from
CRAN at \url{http://CRAN.R-project.org/}.


\bibliography{zoo}

\begin{appendix}
\section{Reference card}
\begin{tabular}{rp{10.5cm}}
\multicolumn{2}{l}{\textbf{Creation}} \\
\code{zoo(x, order.by)} & creation of a \code{"zoo"} object
  from the observations \code{x} (a vector or a matrix) and an index
  \code{order.by} by which the observations are ordered. \\
& For computations on arbitrary index classes, methods to the 
  following genric functions are assumed to work: combining \code{c()},
  querying length \code{length()}, subsetting \code{[}, ordering
  \code{ORDER()} and value matching \code{MATCH()}.\\[0.5cm]

\multicolumn{2}{l}{\textbf{Standard methods}} \\
\code{plot} & plotting \\
\code{lines} & adding a \code{"zoo"} series to a plot \\
\code{print} & printing \\
\code{summary} & summarizing (column-wise) \\
\code{str} & displaying structure of \code{"zoo"} objects \\
\code{head, tail} & head and tail of \code{"zoo"} objects \\[0.5cm]

\multicolumn{2}{l}{\textbf{Coercion}} \\
\code{as.zoo} & coercion to \code{"zoo"} is available for objects
    of class \code{"ts"}, \code{"its"}, \code{"irts"} (plus a default
    method).\\
\code{as.}\textit{class}\code{.zoo} & coercion from \code{"zoo"} to
    other classes. Currently available for \textit{class} in \code{"matrix"},
    \code{"vector"}, \code{"data.frame"}, \code{"list"}, \code{"irts"}
    and \code{"its"}. \\
\code{is.zoo} & querying wether an object is of class \code{"zoo"} \\[0.5cm]

\multicolumn{2}{l}{\textbf{Merging and binding}} \\
\code{merge} & union, intersection, left join, right join along indexes\\
\code{cbind} & column binding along the intersection of the index\\
\code{rbind} & row binding (indexes may not overlap)\\
\code{aggregate} & compute summary statistics along a coarser grid of indexes \\[0.5cm]

\multicolumn{2}{l}{\textbf{Mathematical operations}} \\
\code{Ops} & group generic functions performed along the intersection of indexes\\
\code{t} & transposing (coerces to \code{"matrix"} before) \\
\code{cumsum} & compute (columnwise) cumulative quantities: sums
    \code{cumsum()}, products \code{cumprod()}, maximum \code{cummax()},
    minimum \code{cummin()}.\\[0.5cm]

\multicolumn{2}{l}{\textbf{Extracting and replacing data and index}} \\
\code{index, time} & extract the index of a series\\
\code{index<-}, \code{time<-} & replace the index of a series\\
\code{coredata}, \code{coredata<-} & extract and replace the data associated with a \code{"zoo"} object\\
\code{lag} & lagged observations \\
\code{diff} & arithmetic and geometric differences \\
\code{start, end} & querying start and end of a series \\
\code{window, window<-} & subsetting of \code{"zoo"} objects
    using their index\\[0.5cm]

\multicolumn{2}{l}{\textbf{\code{NA} handling}} \\
\code{na.omit} & omit \code{NA}s \\
\code{na.contiguous} & compute longest sequence of non-\code{NA} observations \\
\code{na.locf} & impute \code{NA}s by carrying forward the last observation\\
\code{na.approx} & impute \code{NA}s by interpolation

%%\\[[0.5cm]]
%%
%% \multicolumn{2}{l}{\textbf{Rolling operations}} \\
%% \code{runmean} & running mean, median and maxim are \code{runmean}, \code{runmed} and
%%   \code{runmax}, respectively
\end{tabular}

\end{appendix}

\end{document}
