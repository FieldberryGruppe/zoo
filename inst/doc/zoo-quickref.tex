\documentclass{Z}
\DeclareGraphicsExtensions{.pdf,.eps}
\newcommand{\mysection}[1]{\subsubsection[#1]{\textbf{#1}}}

%% need no \usepackage{Sweave}

\author{Ajay Shah\\Ministry of Finance, New Delhi \And
        Achim Zeileis\\Wirtschaftsuniversit\"at Wien \And
        Gabor Grothendieck\\GKX Associates Inc.}
\Plainauthor{Ajay Shah, Achim Zeileis, Gabor Grothendieck}

\title{\pkg{zoo} Quick Reference}
\Plaintitle{zoo Quick Reference}

\Keywords{irregular time series, daily data, weekly data, returns}

\Abstract{
  This vignette gives a brief overview of (some of) the functionality contained
  in \pkg{zoo} including several nifty code snippets when dealing
  with (daily) financial data. For a more complete overview of the
  package's functionality and extensibility see 
  \cite{zoo:Zeileis+Grothendieck:2005} (contained as vignette ``zoo'' in the
  package), the manual pages and the reference card.  
}

\begin{document}


%\VignetteIndexEntry{zoo Quick Reference}
%\VignetteDepends{zoo,tseries}
%\VignetteKeywords{irregular time series, daily data, weekly data, returns}
%\VignettePackage{zoo}



\mysection{Read a series from a text file}

To read in data in a text file, \code{read.table()} and associated
functions can 
be used as usual with \code{zoo()} being called subsequently.
The convenience function \code{read.zoo} is a simple wrapper to these
functions that assumes the index is in the first column of the file
and the remaining columns are data.

Data in \code{demo1.txt}, where each row looks like
\begin{verbatim}
   23 Feb 2005|43.72
\end{verbatim}
can be read in via
\begin{Schunk}
\begin{Sinput}
R> inrusd <- read.zoo("demo1.txt", sep = "|", format = "%d %b %Y")
\end{Sinput}
\end{Schunk}
The \code{format} argument causes the first column to be transformed
to an index of class \code{"Date"}.

The data in \code{demo2.txt} look like
\begin{verbatim}
   Daily,24 Feb 2005,2055.30,4337.00
\end{verbatim}
and requires more attention because of the format of
the first column.
\begin{Schunk}
\begin{Sinput}
R> tmp <- read.table("demo2.txt", sep = ",")
R> z <- zoo(tmp[, 3:4], as.Date(as.character(tmp[, 2]), format = "%d %b %Y"))
R> colnames(z) <- c("Nifty", "Junior")
\end{Sinput}
\end{Schunk}

\mysection{Query dates}

To return all dates corresponding to a series
\code{index(z)} or equivalently 
\begin{Schunk}
\begin{Sinput}
R> time(z)
\end{Sinput}
\begin{Soutput}
 [1] "2005-02-10" "2005-02-11" "2005-02-14" "2005-02-15" "2005-02-17"
 [6] "2005-02-18" "2005-02-21" "2005-02-22" "2005-02-23" "2005-02-24"
[11] "2005-02-25" "2005-02-28" "2005-03-01" "2005-03-02" "2005-03-03"
[16] "2005-03-04" "2005-03-07" "2005-03-08" "2005-03-09" "2005-03-10"
\end{Soutput}
\end{Schunk}
can be used. The first and last date can be obtained by
\begin{Schunk}
\begin{Sinput}
R> start(z)
\end{Sinput}
\begin{Soutput}
[1] "2005-02-10"
\end{Soutput}
\begin{Sinput}
R> end(inrusd)
\end{Sinput}
\begin{Soutput}
[1] "2005-03-10"
\end{Soutput}
\end{Schunk}

\mysection{Convert back into a plain matrix}

To strip off the dates and just return a plain vector/matrix
\code{coredata} can be used
\begin{Schunk}
\begin{Sinput}
R> plain <- coredata(z)
R> str(plain)
\end{Sinput}
\begin{Soutput}
 num [1:20, 1:2] 2063 2082 2098 2090 2062 ...
 - attr(*, "dimnames")=List of 2
  ..$ : chr [1:20] "1" "2" "3" "4" ...
  ..$ : chr [1:2] "Nifty" "Junior"
\end{Soutput}
\end{Schunk}

\mysection{Union and intersection}

Unions and intersections of series can be computed by \code{merge}. The
intersection are those days where both series have time points:
\begin{Schunk}
\begin{Sinput}
R> m <- merge(inrusd, z, all = FALSE)
\end{Sinput}
\end{Schunk}
whereas the union uses all dates and fills the gaps where one
series has a time point but the other does not 
with \code{NA}s (by default):
\begin{Schunk}
\begin{Sinput}
R> m <- merge(inrusd, z)
\end{Sinput}
\end{Schunk}

\code{cbind(inrusd, z)} is almost equivalent to the \code{merge}
call, but may lead to inferior naming in some situations 
hence \code{merge} is preferred

To combine a series with its lag, use
\begin{Schunk}
\begin{Sinput}
R> merge(inrusd, lag(inrusd, -1))
\end{Sinput}
\begin{Soutput}
           inrusd lag(inrusd, -1)
2005-02-10  43.78              NA
2005-02-11  43.79           43.78
2005-02-14  43.72           43.79
2005-02-15  43.76           43.72
2005-02-16  43.82           43.76
2005-02-17  43.74           43.82
2005-02-18  43.84           43.74
2005-02-21  43.82           43.84
2005-02-22  43.72           43.82
2005-02-23  43.72           43.72
2005-02-24  43.70           43.72
2005-02-25  43.69           43.70
2005-02-28  43.64           43.69
2005-03-01  43.72           43.64
2005-03-02  43.70           43.72
2005-03-03  43.65           43.70
2005-03-04  43.71           43.65
2005-03-07  43.69           43.71
2005-03-09  43.67           43.69
2005-03-10  43.58           43.67
\end{Soutput}
\end{Schunk}

\mysection{Visualization}

By default, the \code{plot} method generates a graph for each
series in \code{m}
\begin{center}
\setkeys{Gin}{width=0.7\textwidth}
\begin{Schunk}
\begin{Sinput}
R> plot(m)
\end{Sinput}
\end{Schunk}
\includegraphics{zoo-quickref-plotting1}
\end{center}

but several series can also be plotted in a single window.
\begin{center}
\setkeys{Gin}{width=0.7\textwidth}
\begin{Schunk}
\begin{Sinput}
R> plot(m[, 2:3], plot.type = "single", col = c("red", "blue"), 
+     lwd = 2)
\end{Sinput}
\end{Schunk}
\includegraphics{zoo-quickref-plotting2}
\end{center}

\mysection{Select (a few) observations}

Selections can be made for a range of dates of interest
\begin{Schunk}
\begin{Sinput}
R> window(z, start = as.Date("2005-02-15"), end = as.Date("2005-02-28"))
\end{Sinput}
\begin{Soutput}
             Nifty  Junior
2005-02-15 2089.95 4367.25
2005-02-17 2061.90 4320.15
2005-02-18 2055.55 4318.15
2005-02-21 2043.20 4262.25
2005-02-22 2058.40 4326.10
2005-02-23 2057.10 4346.00
2005-02-24 2055.30 4337.00
2005-02-25 2060.90 4305.75
2005-02-28 2103.25 4388.20
\end{Soutput}
\end{Schunk}
and also just for a single date
\begin{Schunk}
\begin{Sinput}
R> m[as.Date("2005-03-10")]
\end{Sinput}
\begin{Soutput}
           inrusd  Nifty  Junior
2005-03-10  43.58 2167.4 4648.05
\end{Soutput}
\end{Schunk}

\mysection{Handle missing data}

Various methods for dealing with \code{NA}s are available, including
linear interpolation
\begin{Schunk}
\begin{Sinput}
R> interpolated <- na.approx(m)
\end{Sinput}
\end{Schunk}
`last observation carried forward',
\begin{Schunk}
\begin{Sinput}
R> m <- na.locf(m)
R> m
\end{Sinput}
\begin{Soutput}
           inrusd   Nifty  Junior
2005-02-10  43.78 2063.35 4379.20
2005-02-11  43.79 2082.05 4382.90
2005-02-14  43.72 2098.25 4391.15
2005-02-15  43.76 2089.95 4367.25
2005-02-16  43.82 2089.95 4367.25
2005-02-17  43.74 2061.90 4320.15
2005-02-18  43.84 2055.55 4318.15
2005-02-21  43.82 2043.20 4262.25
2005-02-22  43.72 2058.40 4326.10
2005-02-23  43.72 2057.10 4346.00
2005-02-24  43.70 2055.30 4337.00
2005-02-25  43.69 2060.90 4305.75
2005-02-28  43.64 2103.25 4388.20
2005-03-01  43.72 2084.40 4382.25
2005-03-02  43.70 2093.25 4470.00
2005-03-03  43.65 2128.85 4515.80
2005-03-04  43.71 2148.15 4549.55
2005-03-07  43.69 2160.10 4618.05
2005-03-08  43.69 2168.95 4666.70
2005-03-09  43.67 2160.80 4623.85
2005-03-10  43.58 2167.40 4648.05
\end{Soutput}
\end{Schunk}
and others.

\mysection{Prices and returns}

To compute log-difference returns in \%, the following
convenience function is defined
\begin{Schunk}
\begin{Sinput}
R> prices2returns <- function(x) 100 * diff(log(x))
\end{Sinput}
\end{Schunk}
which can be used to convert all columns (of prices) into returns.
\begin{Schunk}
\begin{Sinput}
R> r <- prices2returns(m)
\end{Sinput}
\end{Schunk}

A 10-day rolling window standard deviations (for all columns) can
be computed by
\begin{Schunk}
\begin{Sinput}
R> rollapply(r, 10, sd)
\end{Sinput}
\begin{Soutput}
               inrusd     Nifty    Junior
2005-02-17 0.14599121 0.6993355 0.7878843
2005-02-18 0.14527421 0.6300543 0.8083622
2005-02-21 0.14115862 0.8949318 1.0412806
2005-02-22 0.15166883 0.9345299 1.0256508
2005-02-23 0.14285470 0.9454103 1.1957959
2005-02-24 0.13607992 0.9453855 1.1210963
2005-02-25 0.11962991 0.9334899 1.1105966
2005-02-28 0.11963193 0.8585071 0.9388661
2005-03-01 0.09716262 0.8569891 0.9131822
2005-03-02 0.09787943 0.8860388 1.0566389
2005-03-03 0.11568119 0.8659890 1.0176645
\end{Soutput}
\end{Schunk}

To go from a daily series to the series of just the last-traded-day of each month
\code{aggregate} can be used
\begin{Schunk}
\begin{Sinput}
R> prices2returns(aggregate(m, as.yearmon, tail, 1))
\end{Sinput}
\begin{Soutput}
             inrusd    Nifty   Junior
Mar 2005 -0.1375831 3.004453 5.752866
\end{Soutput}
\end{Schunk}

Analogously, the series can be aggregated to the last-traded-day of each week
employing a convenience function \code{nextfri} that computes for each \code{"Date"}
the next friday.
\begin{Schunk}
\begin{Sinput}
R> nextfri <- function(x) 7 * ceiling(as.numeric(x - 5 + 4)/7) + 
+     as.Date(5 - 4)
R> prices2returns(aggregate(na.locf(m), nextfri, tail, 1))
\end{Sinput}
\begin{Soutput}
                inrusd      Nifty     Junior
2005-02-18  0.11411618 -1.2809533 -1.4883536
2005-02-25 -0.34273997  0.2599329 -0.2875731
2005-03-04  0.04576659  4.1464226  5.5076988
2005-03-11 -0.29785794  0.8921286  2.1419450
\end{Soutput}
\end{Schunk}

\mysection{Query Yahoo! Finance}

When connected to the internet, Yahoo! Finance can be easily queried using
the \code{get.hist.quote} function in
\begin{Schunk}
\begin{Sinput}
R> library("tseries")
\end{Sinput}
\end{Schunk}


From version 0.9-30 on, \code{get.hist.quote} by default returns \verb/"zoo"/ series with
a \verb/"Date"/ attribute (in previous versions these had to be transformed from \verb/"ts"/
`by hand').

A daily series can be obtained by:
\begin{Schunk}
\begin{Sinput}
R> sunw <- get.hist.quote(instrument = "SUNW", start = "2004-01-01", 
+     end = "2004-12-31")
\end{Sinput}
\end{Schunk}

A monthly series can be obtained and transformed by
\begin{Schunk}
\begin{Sinput}
R> sunw2 <- get.hist.quote(instrument = "SUNW", start = "2004-01-01", 
+     end = "2004-12-31", compression = "m", quote = "Close")
\end{Sinput}
\end{Schunk}

Here, \verb/"yearmon"/ dates might be even more useful:
\begin{Schunk}
\begin{Sinput}
R> time(sunw2) <- as.yearmon(time(sunw2))
\end{Sinput}
\end{Schunk}

The same series can equivalently be computed from the daily series via
\begin{Schunk}
\begin{Sinput}
R> sunw3 <- aggregate(sunw[, "Close"], as.yearmon, tail, 1)
\end{Sinput}
\end{Schunk}

The corresponding returns can be computed via
\begin{Schunk}
\begin{Sinput}
R> r <- prices2returns(sunw3)
\end{Sinput}
\end{Schunk}
where \code{r} is still a \verb/"zoo"/ series.


\mysection{Query Oanda}

Similarly you can obtain historical exchange rates from \url{http://www.oanda.com/}
using \code{get.hist.quote}.

A daily series of EUR/USD exchange rates can be queried by
\begin{Schunk}
\begin{Sinput}
R> eur.usd <- get.hist.quote(instrument = "EUR/USD", provider = "oanda", 
+     start = "2004-01-01", end = "2004-12-31")
\end{Sinput}
\end{Schunk}

This contains the exchange rates for every day in 2004. However, it is common practice
in many situations to exclude the observations from weekends. To do so, we
write a little convenience function which can determine for a vector of \code{"Date"}
observations whether it is a weekend or not

\begin{Schunk}
\begin{Sinput}
R> is.weekend <- function(x) ((as.numeric(x) - 2)%%7) < 2
\end{Sinput}
\end{Schunk}

Based on this we can omit all observations from weekends
\begin{Schunk}
\begin{Sinput}
R> eur.usd <- eur.usd[!is.weekend(time(eur.usd))]
\end{Sinput}
\end{Schunk}

The function \code{is.weekend} introduced above exploits the fact that a \code{"Date"}
is essentially the number of days since 1970-01-01, a Thursday. A mor intelligible
function which yields identical results could be based on the \code{"POSIXlt"} class

\begin{Schunk}
\begin{Sinput}
R> is.weekend <- function(x) {
+     x <- as.POSIXlt(x)
+     x$wday > 5 | x$wday < 1
+ }
\end{Sinput}
\end{Schunk}

\bibliography{zoo}

\end{document}

